% $Id$

\label{sec:xgrid:offline}
----------------------------------------------------------------------
Current functionality in the ESMF allows for an {\tt ESMF\_XGrid} object to be internally generated 
through the use of ESMF primitive functions and built-in Fortran libraries. This capability has been 
enhanced such that a pre-computed FMS exchange grid can now be handed-off to ESMF in offline mode 
before downstream flux calculations occur. The benefit of this method is to incorporate existing FMS 
files and methodologies in addition to consistency with the FMS XGrid computation. Note that it is 
necessary that all user supplied FMS NetCDF files have complete headers, including descriptions. They 
will otherwise need to be edited.

The stages of this process are first to read in the FMS-generated NetCDF grid files that  
represent the model component grids on either side of the exchange grid. An ESMF {\tt ESMF\_XGrid} 
object is then created from this user supplied information. This {\tt ESMF\_XGrid} contains the 
necessary variables that are not necessarily present in an FMS-generated XGrid. 

Following this, the FMS-generated XGrid NetCDF file can be read using the {\tt ESMF\_FieldBundleRead()} 
method to place the FMS exchange grid file into memory, while {\tt ESMF\_FieldGet()} copies the variables 
and arrays to the local space so information can later be extracted and modified.  {\tt ESMF\_FieldGet()} 
populates the Fortran fields with the actual values and {\tt ESMF\_GridValidate()} ensures the grids and 
meshes properly intersect when retiling is performed.

The remaining step is to remap the XGrid i,j indices from FMS to the ESMF standard linear index. This is
necessary in order to correctly sequence the arrays to the {\tt ESMF\_XGrid}.  The index renumbering
routine added here is designed to take an arbitrary maximum length and width value for any given exchange 
grid.  A one-dimensional scalar array, seqIndexArray, is used to hold the new ESMF index list after the 
grid renumbering takes place.  The program loops through the original grid and assigns the (i,j) index 
pair to a new index.  The new sequence of indices is then printed out to confirm the new array has been 
constructed correctly.  The centroids of the sequential indices were found by calculating the midpoint of 
the neighboring grid cells.  This method loops through all the indices in an array and then calculates the 
middle distance between two adjacent grid cells.  An array was constructed to store the final centroids 
that would be input to the {\tt ESMF\_XGridCreateFromSparseMat()} routine for building the equivalent 
{\tt ESMF\_XGrid}.
